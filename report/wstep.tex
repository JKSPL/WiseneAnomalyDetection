28 stycznia 2006 roku całą Polską wstrząsnęła tragedia — pod naporem śniegu zawaliła się hala MTK, w wyniku czego
zginęło 65 osób.
Po tym wydarzeniu firma WISENE Roof Monitoring opracowała rozwiązanie, które może zapobiec tego
typu katastrofom.
Projekt opiera się na zainstalowaniu szeregu precyzyjnych czujników laserowych, z których pomiaru można wywnioskować stopień ugięcia dachu.
Czujniki mogą być ustawione poziomo pod dachem, pod kątem, lub pionowo, precyzyjnie wykrywając milimetrowe ugięcia dachu.

Siłą przedstawionego rozwiązania jest jego prostota — nie potrzeba skomplikowanego przygotowania, by zainstalować taki system.
Laserowe czujniki są tanie w utrzymaniu, zarówno materiałowo, jak i organizacyjnie.
Natomiast wadą systemu jest jego wrażliwość na możliwe zaburzenia pomiarów odległości.
Przy zainstalowaniu takiego systemu w działających przedsiębiorstwach, nieuniknione jest potencjalne zastawienie czujników.
Zastawienia mogą być bardzo wyraźne — np.\ po ustawieniu dwumetrowej szafy, ale też mogą być delikatne — w wyniku dłuższego pozostawienia paczki papierosów.
%\begin{wrapfigure}{r}{0.25\textwidth}
%    \centering
%    \begin{tikzpicture}[]
%        \path[draw]
%  (0.0, 0.0)
%        -- (0.0, 1.0)
%        (1.0, 1.0)
%        -- (1.0, 0.0)
%        (1.0, 0.0)
%        -- (0.0, 0.0);
%        \path[draw, bend left]	(0.0, 1.0)	to[in=-150, out=-30] (1.0, 1.0);
%    \end{tikzpicture}
%    \caption{Wraz z obciążeniem dachu, maleje odległość czujnika od zadaszenia do podłoża.}
%\end{wrapfigure}

Celem tego projektu jest zażegnanie tego problemu poprzez opracowanie ulepszonego systemu raportującego, czy pomiar czujnika jest zastawiony, czy nie.
Jest to kluczowy komponent systemu, ponieważ jego brak może powodować zbyt częste raportowanie fikcyjnych ugięć, a co za tym idzie, ignorowania raportowanych alarmów.
Ponadto system szacuje prawdopodobną wartość wykrytego zastawienia.